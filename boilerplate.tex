% boilerplate.tex - A boilerplate for the lazy; too lazy indeed to
% comment own boilerplate
\documentclass[a4paper,twosided,11pt,DIV15]{scrartcl}

% nrs.sty: generic collection of packages and macros from some years of 
% experience. It includes elements which are sure to be beneficial and do 
% not alter the overall structure of an average document. All other packages
% are considered optional and listed in the preample below.
\usepackage{nrs}

% Mechanician's friendly little LaTeX package (for math)
\usepackage{msym}

% Fonts
%\usepackage[garamond]{mathdesign}
%\usepackage{times}
%\usepackage{pxfonts}%ccfonts,times,mathpazo,mathpple
\usepackage[T1]{fontenc}

% Search directory for image files
\graphicspath{{./img/}}

% Engineering notation for numbers
\sisetup{output-exponent-marker=\ensuremath{\mathrm{E}}}

% Doc options
\title{Lorem Ipsum Dolor Sit Amet}
\author{J. Doe}

% Misc
%\usepackage{algpseudocode,algorithm}
%\usepackage[small,compact]{titlesec}
%\setlength{\parindent}{0pt}
%\usepackage{parskip}
%\usepackage[cm]{fullpage}
%\usepackage{breqn}

\begin{document}

%\assletterhead
\maketitle
%\tableofcontents


% To clean the boilerplate, 44ggV300ggd




\paragraph{(A1)}
Sum of $\boa$ and $\bob$ can be shown in basis notation as follows,
$$ \boa+\bob=a_i\boe_i + b_i\boe_i = (a_i+b_i)\boe_i$$
and hence is calculated as
$$ = \left(\rvec{3;2}+\rvec{3;-1}\right)\boe_i=\rvec{6;1}\boe_i$$

The vectors $\boc$ and $\bod$ are linearly independent hence form a
basis, and this is proved through

$$\det\begin{bmatrix}c_i\\d_i\end{bmatrix} = 
\begin{vmatrix}2 & -2\\1&2\end{vmatrix} = 6 \neq 0\quad \therefore
\mathrm{linearly\ independent}$$

The decomposition of the sum $\bos:=\boa+\bob$ with respect to basis $\bar{\boe}_i=\rvec{\boc;\bod}$, in other words finding the coefficients $\bar{s}_i$ follows as
$$\bar{s}_i\bar{\boe}_i = s_i\boe_i\quad\Longrightarrow\quad 
%\bar{s}_1\boc + \bar{s}_2\bod = \cvec{6;1}\quad\Longrightarrow\quad
\rvec{\bar{s}_1;\bar{s}_2}\cvec{\boc;\bod}=\cvec{6;1}\quad\Longrightarrow\quad
\rvec{\bar{s}_1;\bar{s}_2}\begin{bmatrix}2&-2\\1&2\end{bmatrix}=\cvec{6;1}
$$
$$\quad\Longrightarrow\quad
\rvec{\bar{s}_1;\bar{s}_2}=\cvec{6;1}\begin{bmatrix}2&-2\\1&2\end{bmatrix}^{-1}
= \rvec{1.833;2.333}
$$
Therefore $\bos = 1.833\boc + 2.333\bod$. Graphical proof is given
below:\\

\paragraph{(A2)} (a)\quad If the vectors $\boa$, $\bob$, $\boc$ are to form a
basis, they should be linearly independent.
$$ \det\cvec{\boa;\bob;\boc} = 
\begin{vmatrix} -1&2&3\\1&-2&3\\1&2&3\end{vmatrix} = 24 \neq 0 
\quad \therefore \mathrm{linearly\ independent}$$

(b)\quad The unary operator $\unit\ \bullet$
unitizes a vector $\bov$ with $\unit\ \bov = \frac{\bov}{\|\bov\|}$.
\begin{align*}
\bar{\boa} &= \unit\ \boa = \rvec{-0.2672;0.5345;0.8017}\\
\bar{\bob} &= \unit\ \boa = \rvec{0.2672;-0.5345;0.8017}\\ 
\bar{\boc} &= \unit\ \boc = \rvec{0.2672;0.5345;0.8017}
\end{align*}

(c)\quad Naming the newly formed orthonormal basis as
$\bar{\boe}_i=\rvec{\bar{\boa};\bar{\bob};\bar{\boc}}$, we could obtain
the coefficients of $\bod$, $\hat{d}_i$ which are with respect to
basis $\boe_i$. Basically,
$$d_i\bar{\boe}_i = \hat{d}_i\boe_i$$
$$\Longrightarrow\ 
\hat{d}_i\boe_i=\rvec{4;3;1}\cdot
\begin{bmatrix}
-0.2672&0.5345&0.8017\\
0.2672&-0.5345&0.8017\\
0.2672&0.5345&0.8017\\
\end{bmatrix}\boe_i
=\rvec{0.;1.069;6.414}\boe_i
$$
(d)\quad In order to calculate the volume of the parallelepiped
spanned by $\boa$, $\bob$, $\boc$ the parallelepidial product of the
three vectors can be taken:
$$ (\boa \times \bob)\cdot \boc = \det\cvec{\boa;\bob;\boc} = 
\begin{vmatrix} -1&2&3\\1&-2&3\\1&2&3\end{vmatrix} = 24$$
Hence the volume of the parallelepiped is 24.
\paragraph{(A3)}
For $i,\ j,\ k,\ l,\ m = 1,\ 2,\ 3$, the expressions below evaluate as
\begin{enumerate}[(a)]
\item $a_i = \rvec{a_1;a_2;a_3}$
\item $A_{ij}=\begin{bmatrix}A_{11}&A_{12}&A_{13}\\A_{21}&A_{22}&A_{23}\\A_{31}&A_{32}&A_{33}\end{bmatrix}$
\item $v_i\boe_i=v_1\boe_1+v_2\boe_2+v_3\boe_3$

\item $A_{ij}v_i = \cvec{A_{11}v_1+A_{21}v_2+A_{31}v_3;A_{12}v_1+A_{22}v_2+A_{32}v_3;A_{13}v_1+A_{23}v_2+A_{33}v_3}$

\item $a_i\delta_{ik} = a_k = \rvec{a_1;a_2;a_3}$
\item $A_{ki}\delta_{il} = A_{kl} = \begin{bmatrix}A_{11}&A_{12}&A_{13}\\A_{21}&A_{22}&A_{23}\\A_{31}&A_{32}&A_{33}\end{bmatrix}$
\item $A_{ij}B_{ij} = A_{11}B_{11}+A_{12}B_{12}+A_{13}B_{13}+A_{21}B_{21}+A_{22}B_{22}+A_{23}B_{23}+A_{31}B_{31}+A_{32}B_{32}+A_{33}B_{33} $
\item $A_{im}B_{kl}\delta_{jm}\delta_{jk} = A_{ij}B_{jl}$\\
\begin{align*}
&=
\begin{bmatrix}
A_{11}B_{11}  & A_{11}B_{12}  & A_{11}B_{13} \\
A_{21}B_{11}  & A_{21}B_{12}  & A_{21}B_{13} \\
A_{31}B_{11}  & A_{31}B_{12}  & A_{31}B_{13} \\
\end{bmatrix}+
\begin{bmatrix}
A_{12}B_{21}  & A_{12}B_{22}  & A_{12}B_{23} \\
A_{22}B_{21}  & A_{22}B_{22}  & A_{22}B_{23} \\
A_{32}B_{21}  & A_{32}B_{22}  & A_{32}B_{23} \\
\end{bmatrix}+
\begin{bmatrix}
A_{13}B_{31}  & A_{13}B_{32}  & A_{13}B_{33} \\
A_{23}B_{31}  & A_{23}B_{32}  & A_{23}B_{33} \\
A_{33}B_{31}  & A_{33}B_{32}  & A_{33}B_{33} \\
\end{bmatrix}\\
&=
\begin{bmatrix}
A_{11}B_{11}  + A_{12}B_{21}  + A_{13}B_{31}  & A_{11}B_{12}  + A_{12}B_{22}  + A_{13}B_{32}  & A_{11}B_{13}  + A_{12}B_{23}  + A_{13}B_{33} \\
A_{21}B_{11}  + A_{22}B_{21}  + A_{23}B_{31}  & A_{21}B_{12}  + A_{22}B_{22}  + A_{23}B_{32}  & A_{21}B_{13}  + A_{22}B_{23}  + A_{23}B_{33} \\
A_{31}B_{11}  + A_{32}B_{21}  + A_{33}B_{31}  & A_{31}B_{12}  + A_{32}B_{22}  + A_{33}B_{32}  & A_{31}B_{13}  + A_{32}B_{23}  + A_{33}B_{33} \\
\end{bmatrix}
\end{align*}

\item $A_{1lm}B_{lm} = A_{111}B_{11}+A_{112}B_{12}+A_{113}B_{13}+A_{121}B_{21}+A_{122}B_{22}+A_{123}B_{23}+A_{131}B_{31}+A_{132}B_{32}+A_{133}B_{33}$
\end{enumerate}

\paragraph{(A4)} If $\Bot$ is a symmetric tensor, $\Bot=\Bot^T=\boa
\otimes \bob = \bob \otimes \boa$. Writing in basis notation,
$$a_ib_k(\boe_i\otimes\boe_k) = a_kb_i(\boe_i\otimes\boe_k)$$
It follows that $a_ib_k=a_kb_i$. The terms relevant to the solution of
this problem are:
\begin{align*}
  a_1b_2 &= a_2b_1 \Rightarrow b_2=\alpha=\frac{a_2b_1}{a_1} =
  \frac{4\times 3}{1} = 12\\
  a_1b_3 &= a_3b_1 \Rightarrow b_3=\beta=\frac{a_3b_1}{a_1} =
  \frac{6\times 3}{1} = 18
\end{align*}

\paragraph{(A5)} Given are the tensors $\Boa=a_{ij}\,\basis{ij}$ and
$\Bob=b_{kl}\,\basis{kl}$.
\begin{enumerate}[(a)]
\item Prove that $\Boa\,\Bob\neq\Bob\,\Boa$.\\
\begin{minipage}{0.5\textwidth}
\begin{align*}
    \Boa\,\Bob &= a_{ij}\,(\basis{ij})\,b_{kl}\,(\basis{kl})\\
    &= a_{ij}\,b_{kl}\,(\basis{ij})\,(\basis{kl}) \\
    &= a_{ij}\,b_{kl}\,\delta_{jk}\,(\basis{il})\\
    &= a_{ij}\,b_{jl}\,(\basis{il})
  \end{align*}
\end{minipage}
\begin{minipage}{0.5\textwidth}
\begin{align*}
    \Bob\,\Boa &= b_{kl}\,(\basis{kl})\,a_{ij}\,(\basis{ij})\\
    &= b_{kl}\,a_{ij}\,(\basis{kl})\,(\basis{ij}) \\
    &= b_{kl}\,a_{ij}\,\delta_{li}\,(\basis{kj})\\
    &= b_{kl}\,a_{lj}\,(\basis{kj})
  \end{align*}
\end{minipage}
\\[1em]
$\therefore a_{ij}\,b_{jl} = b_{kl}\,a_{lj}$ if and only if the
product $\Boa\,\Bob$ is symmetric. Since this is not a general case,
the tensor product of tensors can be said to be \textbf{not} commutative.
\item Prove that $(\Boa\,\Bob)^T=\Bob^T\,\Boa^T$.

\begin{minipage}{0.5\textwidth}
\begin{align*}
    (\Boa\,\Bob)^T &= [a_{ij}\,(\basis{ij})\,b_{kl}\,(\basis{kl})]^T\\
    &= [a_{ij}\,b_{kl}\,(\basis{ij})\,(\basis{kl})]^T \\
    &= [a_{ij}\,b_{kl}\,\delta_{jk}\,(\basis{il})]^T\\
    &= [a_{ij}\,b_{jl}\,(\basis{il})]^T\\
    &= a_{ij}\,b_{jl}\,(\basis{li})
  \end{align*}
\end{minipage}
\begin{minipage}{0.5\textwidth}
\begin{align*}
    \Bob^T\,\Boa^T &= b_{kl}\,(\basis{lk})\,a_{ij}\,(\basis{ji})\\
    &= b_{kl}\,a_{ij}\,(\basis{lk})\,(\basis{ji}) \\
    &= b_{kl}\,a_{ij}\,\delta_{kj}\,(\basis{li})\\
    &= b_{jl}\,a_{ij}\,(\basis{li}) \\
    &= a_{ij}\,b_{jl}\,(\basis{li})
  \end{align*}
\end{minipage}
\\[1em] Hence $(\Boa\,\Bob)^T=\Bob^T\,\Boa^T$.
\\Prove that $\tr(\Boa\,\Bob)=\tr(\Bob\,\Boa)$. By definition,
$\tr(\Boa)=\Boa\cdot \Boi$.\\
\begin{minipage}{0.5\textwidth}
\begin{align*}
    (\Boa\,\Bob)\cdot \Boi &= a_{ij}\,(\basis{ij})\,b_{kl}\,(\basis{kl})\cdot(\basis{mm})\\
    &= a_{ij}\,b_{kl}\,(\basis{ij})\,(\basis{kl})\cdot(\basis{mm}) \\
    &= a_{ij}\,b_{kl}\,\delta_{jk}\,(\basis{il})\cdot(\basis{mm})\\
    &= a_{ij}\,b_{kl}\,\delta_{jk}\,\delta_{im}\,\delta_{lm}\\
    &= a_{ik}\,b_{ki}
  \end{align*}
\end{minipage}
\begin{minipage}{0.5\textwidth}
\begin{align*}
    (\Bob\,\Boa)\cdot \Boi &= b_{kl}\,(\basis{kl})\,a_{ij}\,(\basis{ij})\cdot(\basis{mm})\\
    &= b_{kl}\,a_{ij}\,(\basis{kl})\,(\basis{ij})\cdot(\basis{mm}) \\
    &= b_{kl}\,a_{ij}\,\delta_{li}\,(\basis{kj})\cdot(\basis{mm})\\
    &= b_{kl}\,a_{ij}\,\delta_{li}\,\delta_{km}\,\delta_{jm}\\
    &= b_{ki}\,a_{ik}
  \end{align*}
\end{minipage}
\\[1em] $a_{ik}\,b_{ki}=b_{ki}\,a_{ik}$, hence $\tr(\Boa\,\Bob)=\tr(\Bob\,\Boa)$.
\item The scalar product $\Boa\cdot\Bob$ is equal to
\begin{align*}
  \Boa\cdot\Bob&=a_{ij}\,(\basis{ij})\cdot b_{kl}\,(\basis{kl})  \\
  &=a_{ij}\,b_{kl}\,(\basis{ij})\cdot(\basis{kl})  \\
  &=a_{ij}\,b_{kl}\,\delta_{ik}\,\delta{jl} \\
  &=a_{ij}\,b_{ij}
\end{align*}
Hence the scalar product is equal to
$$a_{ij}b_{ij}=1\times 1+0\times 3+0\times 1+4\times -2+-2\times 3+5\times 1+1\times -2+1\times 4+1\times 2=-4$$
\end{enumerate}

\newcommand{\boea}{\overset{*}{\boe}}

\paragraph{(A6)} The basis system $\boe_i$ is rotated around the
$\boe_3$-axis with the angle $\varphi=30\degree$.

\begin{enumerate}[(a)]
\item  Specify the basis vectors $\boea_i$ of the rotated system.
  \begin{align*}
    \boea_1 &= \cos(30)\,\boe_1 + \sin(30)\,\boe_2\\
    \boea_2 &= -\sin(30)\,\boe_1 + \cos(30)\,\boe_2\\
    \boea_3 &= \boe_3
  \end{align*}
\item Determine the rotation tensor $\Bor$ of the transformation
  $\boea_i=R_{ij}\,\boe_j$.\\
Continuing the previous derivation, the transformation can be
converted into a system with $R_{ij}\,\basis{ij}$ defined as the rotation tensor:
$$ R_{ij} = 
\begin{bmatrix}
  \cos(30) & -\sin(30) & 0 \\
  \sin(30) & \cos(30) & 0 \\
  0 & 0 & 1
\end{bmatrix}
$$
\item Show that the rotation tensor $\Bor$ is orthogonal.
\\ The tensor $\Bor$ is orthogonal if and only if $\Bor^T\,\Bor = \Boi$. Hence
it can be proved by
\begin{align*}
\Bor^T\,\Bor&=
\begin{bmatrix}
  \cos(30) & -\sin(30) & 0 \\
  \sin(30) & \cos(30) & 0 \\
  0 & 0 & 1
\end{bmatrix}^T
\begin{bmatrix}
  \cos(30) & -\sin(30) & 0 \\
  \sin(30) & \cos(30) & 0 \\
  0 & 0 & 1
\end{bmatrix}\\
&=
\begin{bmatrix}
  \cos(30) & \sin(30) & 0 \\
  -\sin(30) & \cos(30) & 0 \\
  0 & 0 & 1
\end{bmatrix}
\begin{bmatrix}
  \cos(30) & -\sin(30) & 0 \\
  \sin(30) & \cos(30) & 0 \\
  0 & 0 & 1
\end{bmatrix}
\\&=
\begin{bmatrix}
  1 & 0 & 0\\
  0 & 1 & 0\\
  0 & 0 & 1
\end{bmatrix} \basis{ij}\\
& \therefore \textrm{orthogonal}
\end{align*}

\end{enumerate}



% \renewcommand{\bibname}{References}
% \begin{thebibliography}{9}
% \bibitem{1}
%  J. Grandy 
%   \emph{Efficient Computation of Volume of Hexahedral Cells} 1997\\

% \bibitem{2}
%   T. H. Cormen, C. E. Leiserson, R. L. Rivest, C. Stein
% 	\emph{Introduction to Algorithms 3\textsuperscript{rd}ed.}\\
% \bibitem{3}
%   K. I. Joy 
%   \emph{Coordinate Conversion between the Cartesian Frame
%     and an Arbitrary Frame} 1999

% \end{thebibliography}

\end{document}
